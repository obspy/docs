\documentclass[utf8, compress, hyperref={pdftex=true,pdfpagemode=FullScreen}, graphicx={pdftex}]{beamer}
% for handout add ,handout to []of documentclass
% to serve contents use python -m SimpleHTTPServer 8000

\mode<beamer>
	{
	\useoutertheme{split}
    \usepackage{beamerthemesplit}
	\usecolortheme{lily} % leave this on, sets bg for title
	\usecolortheme[RGB={187,0,0}]{structure} % DARK RED
	%\setbeamercolor{alerted text}{use={structure},fg=structure.fg}
	\useinnertheme{default}
	\setbeamercovered{dynamic} % shows items in white grey before active
  \logo{\includegraphics[height=0.7cm]{obspy-logo.png}}
	%\setbeamertemplate{footline}[frame number]
}

\mode<handout>
	{
	\beamertemplatesolidbackgroundcolor{white}
	\setbeamercolor{alerted text}{fg=black}
	\setbeamercolor{frametitle}{fg=black}
	\setbeamercolor{item}{fg=black}
}

\usepackage{listings}	% syntax highlighting
\usepackage{upquote}    % to force Latex not substitute hyphens
\usepackage{tikz}
\definecolor{mygray}{RGB}{235,241,241}
\definecolor{myred}{RGB}{187,0,0}
\definecolor{mygreen}{RGB}{0,187,0}
\definecolor{myblue}{RGB}{0,0,187}
%\usepackage{courier}
%\usepackage{lmodern} % nice fonts

\hypersetup{pdfpagemode=FullScreen}
\hypersetup{breaklinks=myblue}
\hypersetup{colorlinks=myblue}
\hypersetup{urlcolor=myblue}

\title[Using Python as Glue: Python and C]{\parbox[c][][c]{0.7\paperwidth}{\centering Using Python as Glue: Python and C}}
\author[Moritz Beyreuther]{Moritz Beyreuther}
\date[2012]{Z\"urich, Sept 7 2012}
\institute{ObsPy Workshop at SED/ETH}

\linespread{1.0}

\begin{document}
\lstset{%
language=Python,          % choose the language of the code
keywordstyle=[1]\color{mygreen}\bfseries,
keywordstyle=[2]\color{myblue}\bfseries,
keywordstyle=[3]\color{yellow}\bfseries,
stringstyle=\color{myred},
commentstyle=\color{myblue}\textit,
basicstyle=\ttfamily\footnotesize, % the size of the fonts that are used for the code
numbers=none,             % where to put the line-numbers
numberstyle=\ttfamily\footnotesize,% the size of the fonts that are used for the line-numbers
stepnumber=1,             % step between line-numbers. For 1 each line will be numbered
numbersep=5pt,            % how far the line-numbers are from the code
backgroundcolor=\color{white}, % choose the background color.
showspaces=false,         % show spaces adding particular underscores
showstringspaces=false,   % underline spaces within strings
showtabs=false,           % show tabs within strings adding particular underscores
frame=none,               % e.g. single, adds a frame around the code
tabsize=2,                % sets default tabsize to 2 spaces
captionpos=b,             % sets the caption-position to bottom
breaklines=true,          % sets automatic line breaking
breakatwhitespace=false,  % sets if automatic breaks should only happen at whitespace
escapeinside={\%*}{*)},   % if you want to add a comment within your code
columns=fullflexible      % make PDF pastable
}
\section{title}
\maketitle

\section{motivation}
\begin{frame}
    \frametitle{"for" loops are slow in intepreter languages like python?}
    \begin{description}
        \item[-] missing type system
        \item[-] each iteration checks idx vs size etc.
        \item[solution a:] numpy, use of array programming (for loops are
            compiled in)
        \item[solution b:] access C functions directly
    \end{description}
\end{frame}

\begin{frame}
    \frametitle{pythons foreign function interface ctypes}
    \begin{itemize}
        \item[+] reuse well established C or fortran libraries
        \item[+] direct call C functions in shared libraries
        \item[+] no source code needed
        \item[+] data can be passed directly as pointers
        \item[+] no meta language, only C and Python
        \item[-] python call overhead
    \end{itemize}
\end{frame}

\section{hands on ctypes}
\subsection[basics]{ctypes basic}
\begin{frame}[fragile]
    \frametitle{ctypes basic}
    \begin{lstlisting}
# direct call to libc
import ctypes as C

libc = C.CDLL('libc.so.6')
stamp = libc.time(None)

# print date
import datetime
print datetime.datetime.fromtimestamp(stamp)

libc.printf("hello world")
    \end{lstlisting}
\end{frame}

\subsection[C]{C prototype}
\begin{frame}[fragile]
    \frametitle{C prototype}
    \begin{lstlisting}
typedef struct _headS {
    int N;
    int Nsta;
    int Nlta;
} headS;

int stalta(headS *head, double *data, double *charfct);
    \end{lstlisting}
    \vfill
    \textit{full src code in the appendix}
\end{frame}

\subsection[wrapper]{python wrapper}
\label{wrap}
\begin{frame}[fragile]
    \frametitle{ctypes-wrapper: header}
    \begin{lstlisting}
import numpy as np, ctypes as C

head_t = np.dtype([      # numpy stuctured array
   ('N', 'u4', 1),
   ('nsta', 'u4', 1),
   ('nlta', 'u4', 1),
], align=True)           # align to C-struct border size
data_t = np.dtype('f8')

lib = C.CDLL("./stalta.so")
kwargs = dict(ndim=1, flags='C_CONTIGUOUS')  # one block
lib.stalta.argtypes = [
    np.ctypeslib.ndpointer(dtype=head_t, **kwargs),
    np.ctypeslib.ndpointer(dtype=data_t, **kwargs),
    np.ctypeslib.ndpointer(dtype=data_t, **kwargs),
]
lib.stalta.restype = C.c_int
    \end{lstlisting}
\end{frame}

\begin{frame}[fragile]
    \frametitle{ctypes-wrapper: body}
    \begin{lstlisting}
def stalta(data, nsta, nlta):
    # initialize C struct / numpy structed array
    head = np.empty(1, dtype=head_t)
    head[:] = (len(data), nsta, nlta)
    # ensure correct type and countiguous of data
    data = np.require(data, dtype=data_t,
                      requirements=['C_CONTIGUOUS'])
    # all memory should be allocated by python
    charfct = np.empty(len(data), dtype=data_t)
    # run and check the errorcode
    err = lib.stalta(head, data, charfct)
    if err != 0:
        raise Exception('Error stalta %d' % err)
    return charfct
    \end{lstlisting}
\end{frame}


\subsection{gdb}
\begin{frame}[fragile]
    \frametitle{debugging with gdb}
    \textbf{from the shell:}
    \begin{lstlisting}
export MALLOC_CHECK_=2        # be more strict with memory
gdb $(which python)
    \end{lstlisting}

    \textbf{from within gdb}:
    \begin{lstlisting}
(gdb) # run once to load shared lib symbol table
(gdb) run wrap_stalta.py
(gdb) # set conditional breakpoint and access variables
(gdb) b stalta.c:31
(gdb) info b
(gdb) condition 1 i==4000
(gdb) run wrap_stalta.py
(gdb) print charfct[3995]@10
(gdb) dump binary value head.dump *head
(gdb) dump binary memory data.dump data data+head->N
    \end{lstlisting}
\end{frame}

\subsection{structured arrays}
\begin{frame}[fragile]
    \frametitle{numpy io and structured arrays}
    \begin{lstlisting}
# numpy is great for binary file IO
# loading binary data, < little endian
d = np.fromfile("data.dump", '<f8')
h = np.fromfile("head.dump", dtype=head_t)
print h

u = h.view('u4')  # views are unions
u[1] = 20
print h

# structured arrays vs dictionaries
h = np.zeros(100, dtype=head_t)  # save space
h[:]['N'] = 50                   # supports slicing
print h
    \end{lstlisting}
\end{frame}

\subsection{pdb}
\begin{frame}[fragile]
    \frametitle{python debugger}
    \textbf{set breakpoints in src file via... and run script:}
    \begin{lstlisting}
import pdb; pdb.set_trace()  # standard debugger
# ipython debugger with tab completion
from IPython.core.debugger import Tracer; Tracer()()
# open ipython shell, use then %pylab to load --pylab
from IPython import embed; embed()
    \end{lstlisting}
    \textbf{run script with debugger enabled:}
    \begin{lstlisting}
python -m pdb wrap_stalta.py
ipython -c "%run -d -b25 wrap_stalta.py"  # type c to goto main
    \end{lstlisting}

    \textbf{selected debug commands:}
    \begin{lstlisting}
(pdb) !head['N'] = 5000
(pdb) print head
(pdb) j 22
(pdb) c
    \end{lstlisting}
\end{frame}

\subsection{profile}
\begin{frame}[fragile]
    \frametitle{before optimization: profile!}
    \textbf{either create a profile script:}
    \begin{lstlisting}
from obspy.core import read
from obspy.signal import classicSTALTA
from wrap_stalta import stalta
import cProfile
import pstats

tr = read("/path/to/loc_RJOB20050831023349.z")[0]

cProfile.run("classicSTALTA(tr.data, 400, 2000)", "Profile.prof")
#cProfile.run("stalta(tr.data, 400, 2000)", "Profile.prof")

s = pstats.Stats("Profile.prof")
s.strip_dirs().sort_stats("time").print_stats()
    \end{lstlisting}
\end{frame}

\begin{frame}[fragile]
    \frametitle{before optimization: profile!}
    \textbf{run profiler on whole script:}
    \begin{lstlisting}
python -m cProfile -s cumulative wrap_stalta.py
ipython -c "%run -p -s cumulative wrap_stalta.py"
    \end{lstlisting}
\end{frame}

\section{summary}
\begin{frame}[fragile]
    \frametitle{further ctypes examples}
    obspy uses ctypes as foreign function interface, including callback
    functions, memory resizing, multi-platform (linux, osx, win) ...
    \begin{itemize}
    \item \url{http://obspy.org/browser/obspy/trunk/obspy.gse2/obspy/gse2}
    \item ... obspy/signal
    \item ... obspy/mseed
    \end{itemize}
\end{frame}

\begin{frame}[fragile]
    \frametitle{alternatives}
    \begin{description}
        \item[cython:] python to C compiler
        \item[f2py:] fortran interface generator
        \item[pypy:] reimplementation of python for jit compiling
    \end{description}
\end{frame}

\begin{frame}[fragile]
    \frametitle{future?: from numba.decorators import jit}
    \begin{itemize}
        \item numpy has type info $\rightarrow$ jit is possible.
        \item goal: lightweight JIT, optimize ONLY slow code.
        \item will get ctypes aware.
        \item \url{http://jakevdp.github.com/blog/2012/08/24/numba-vs-cython}
        \item \url{https://github.com/numba/numba}
        \item \url{http://www.drdobbs.com/architecture-and-design/240001128}
    \end{itemize}
\end{frame}

\begin{frame}[fragile]
    \begin{center}
    \Huge{the end}
    \end{center}
\end{frame}

\appendix
\section{appendix}

\begin{frame}[fragile]
    \begin{center}
    \Huge{appendix}
    \end{center}
\end{frame}

\subsection{stalta.c}
\begin{frame}[fragile]
    \frametitle{stalta.c part 1}
    \textbf{compile commands:}
    \begin{lstlisting}
gcc -g -fPIC -O0 -Wall -c stalta.c
gcc -g -shared -Wall -o stalta.so stalta.o
    \end{lstlisting}
    \textbf{stalta.c:}
    \begin{lstlisting}[basicstyle=\ttfamily\tiny]
#include <math.h>

typedef struct _headS {
    int N;
    int Nsta;
    int Nlta;
} headS;
    \end{lstlisting}
\end{frame}

\begin{frame}[fragile]
    \frametitle{stalta.c part 2}
    \begin{lstlisting}[basicstyle=\ttfamily\tiny]
int stalta(headS *head, double *data, double *charfct)
{
    int i, j;
    if (head->N < head->Nlta) {
        return 1;
    }
    for (i = 0; i < head->Nlta; ++i) {
        charfct[i] = 0.0;
    }
    for (i = head->Nlta; i < head->N; ++i) {
        double sta = 0.0;
        double lta;

        for (j = 0u; j < head->Nsta; j++) {
            sta += pow(data[i - j], 2);
        }
        lta = sta;
        for (j = head->Nsta; j < head->Nlta; j++) {
            lta += pow(data[i - j], 2);
        }
        sta /= (double) head->Nsta;
        lta /= (double) head->Nlta;
        charfct[i] = sta / lta;
    }
    return 0;
}
    \end{lstlisting}
\end{frame}

\subsection{ctypes-wrapper: main}
\begin{frame}[fragile]
    \frametitle{\_\_main\_\_ of python wrapper}
    \textit{file wrap\_stalta.py consist of the two parts
    given in the wrapper section (link \ref{wrap}) and
    the following main:}
    \begin{lstlisting}
if __name__ == '__main__':
    from obspy.core import read
    import matplotlib.pyplot as plt

    tr = read("/path/to/loc_RJOB20050831023349.z")[0]
    charfct = stalta(tr.data, 400, 2000)
    
    plt.plot(tr.data)
    plt.twinx().plot(charfct, 'r')
    plt.show()
    \end{lstlisting}
\end{frame}

\end{document}
