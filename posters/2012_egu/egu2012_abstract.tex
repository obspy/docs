\PosterHead{\textbf{\LARGE \huge ObsPy: A Python Toolbox for Seismology/Seismological Observatories}\\[0.5ex]
\large Tobias Megies$^{1}$, Robert Barsch$^{1}$, Moritz Beyreuther, Lion Krischer$^{1}$, Joachim Wassermann$^{1}$ and The ObsPy Development Team\\
$^1$ Department of Earth and Environmental Sciences, Ludwig-Maximilians-Universit\"at M\"unchen\\
Contact:\hspace{0.6cm} \textit{devs@obspy.org}\hspace{0.6cm} \textbf{http://www.obspy.org}}

Python combines the possibilities of a full-blown programming language with the flexibility of an interactive scripting language. Its extensive standard library and many freely available high quality scientific modules cover most needs in developing scientific processing workflows.
ObsPy extends Python's capabilities to fit the specific needs that arise when working with seismological data. It a) comes with a continuously growing signal processing toolbox that covers the most common tasks in seismological analysis, b) provides read and write support for many common waveform and metadata file formats and c) enables access to various data centers, webservices and databases to retrieve waveform data and station/event metadata.
In combination with widely used, free Python packages like NumPy, SciPy, Matplotlib, IPython and PyQt, ObsPy makes it possible to develop complete workflows in Python, ranging from reading locally stored data or requesting data from one or more different data centers via signal analysis and data processing to visualization in GUI applications, output of modified/derived data and creating publication-quality figures.
All functionality is extensively documented and the ObsPy Gallery/Tutorial give a good impression of the wide range of use cases. ObsPy is tested and running on Linux, MacOSX and Windows XP/Vista/7 and comes with installation routines for these systems. ObsPy is developed in a test-driven approach and is available under the GPL/LGPLv3 licences.
Users are welcome to request help, report bugs or propose enhancements via the user mailing list or the Trac ticket system.
